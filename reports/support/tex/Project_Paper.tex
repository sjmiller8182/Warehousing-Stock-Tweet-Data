
%% bare_jrnl.tex
%% V1.4b
%% 2015/08/26
%% by Michael Shell
%% see http://www.michaelshell.org/
%% for current contact information.
%%
%% This is a skeleton file demonstrating the use of IEEEtran.cls
%% (requires IEEEtran.cls version 1.8b or later) with an IEEE
%% journal paper.
%%
%% Support sites:
%% http://www.michaelshell.org/tex/ieeetran/
%% http://www.ctan.org/pkg/ieeetran
%% and
%% http://www.ieee.org/

%%*************************************************************************
%% Legal Notice:
%% This code is offered as-is without any warranty either expressed or
%% implied; without even the implied warranty of MERCHANTABILITY or
%% FITNESS FOR A PARTICULAR PURPOSE! 
%% User assumes all risk.
%% In no event shall the IEEE or any contributor to this code be liable for
%% any damages or losses, including, but not limited to, incidental,
%% consequential, or any other damages, resulting from the use or misuse
%% of any information contained here.
%%
%% All comments are the opinions of their respective authors and are not
%% necessarily endorsed by the IEEE.
%%
%% This work is distributed under the LaTeX Project Public License (LPPL)
%% ( http://www.latex-project.org/ ) version 1.3, and may be freely used,
%% distributed and modified. A copy of the LPPL, version 1.3, is included
%% in the base LaTeX documentation of all distributions of LaTeX released
%% 2003/12/01 or later.
%% Retain all contribution notices and credits.
%% ** Modified files should be clearly indicated as such, including  **
%% ** renaming them and changing author support contact information. **
%%*************************************************************************


% *** Authors should verify (and, if needed, correct) their LaTeX system  ***
% *** with the testflow diagnostic prior to trusting their LaTeX platform ***
% *** with production work. The IEEE's font choices and paper sizes can   ***
% *** trigger bugs that do not appear when using other class files.       ***                          ***
% The testflow support page is at:
% http://www.michaelshell.org/tex/testflow/

% formatting instructions
% https://ras.papercept.net/conferences/support/files/IEEEtran_HOWTO.pdf

% from here
%https://journals.ieeeauthorcenter.ieee.org/create-your-ieee-journal-article/authoring-tools-and-templates/ieee-article-templates/templates-for-transactions/

\documentclass[journal]{IEEEtran}

\usepackage{cite}
\usepackage{amsmath}
\usepackage{url}
\usepackage{multirow}
% don't use the borders around links
\usepackage[hidelinks]{hyperref}

% *** GRAPHICS RELATED PACKAGES ***
%
\ifCLASSINFOpdf
  \usepackage[pdftex]{graphicx}
  % declare the path(s) where your graphic files are
\graphicspath{{../images/}}
  % and their extensions so you won't have to specify these with
  % every instance of \includegraphics
  % \DeclareGraphicsExtensions{.pdf,.jpeg,.png}
\else
  % or other class option (dvipsone, dvipdf, if not using dvips). graphicx
  % will default to the driver specified in the system graphics.cfg if no
  % driver is specified.
  % \usepackage[dvips]{graphicx}
  % declare the path(s) where your graphic files are
  % \graphicspath{{../eps/}}
  % and their extensions so you won't have to specify these with
  % every instance of \includegraphics
  % \DeclareGraphicsExtensions{.eps}
\fi

\begin{document}

\title{Big Data Lake Solution for\\ 
	Warehousing Stock Data and Tweet Data}

\author{Paul~Adams, paula@smu.edu,
        Rikel~Djoko, rdjoko@smu.edu,
        and Stuart Miller, stuart@smu.edu}% <-this % stops a space

% The paper headers
\markboth{MSDS7330 Term Paper - First Draft}
{MSDS7330 Term Paper - First Draft}
% The only time the second header will appear is for the odd numbered pages
% after the title page when using the twoside option.

% make the title area
\maketitle

% This should be about 250 words
\begin{abstract}

Purpose - This research paper aims to discover an optimal solution for parallel management and processing of financial markets data into  warehousing and analysis engines used for buy-sell decision-making. Methods analyzed herein are components of the Hadoop ecosystem. Included is an in-depth analysis of parallel processing - using the Elastic MapReduce package on Amazon Web Services - and data warehousing with Apache Hive. Apache NiFi is used to direct workflow automation for data migration into the warehouse. Finally, a complete assessment of the combinations of levels of the various Hadoop ecosystem applications used is provided in the context of statistical inference.

Design, Methodology, and Approach - One pertinent, underlying hypothesis within this study is to prove that there are differences in processing speeds between S3 and HDFS using normalized and optimized schema designs across multiple MapReduce configurations. This analysis is performed using 100 samples of the same volume of data in a repeated measures analysis using a Hotelling-T statistic. The two highest-performing configurations of S3 and HDFS are then assessed. (We need to get this information and report it) for the next section.

Findings - EXAMPLE: Applying S3 with an optimized schema using 10 reduces, map memory allocation of 2,048 mb and a reduce memory allocation of 4,096 mb is optimal for a more expensive S3 approach. Using HDFS from local storage with a configuration of 20 reduces, map memory allocation of 8,096 megabytes and reduce memory allocation of 10,020 megabytes is ideal for batch-level migrations and querying for large-volume processing. Across n repeated measures, using a one-tailed alpha, the resulting p-value is significant at `Pr > |t|` < x.xxxx (confidence interval (x1, x2)), indicating local storage from HDFS outperforms S3 when using the selected configurations. However, local data storage capacity does not scale well for HDFS compared to cloud-based S3.

\end{abstract}

\section{Introduction}

\IEEEPARstart{D}{ata} in the twenty-first century is expanding in volumes
 at exponential rates; every additional source of data that can act as a
 medium for data communication can obtain useful information, which,
 with modern technology, can be structured and stored, accessible to any
 who have the skills and need to make use of it \cite{BigDataComputing}. 
This information is increasingly profitable. 
However, with the increasing ability to capture and store data from many
 disparate sources, the need to store larger volumes of the data is
 likewise an increasing issue when it comes time to access and apply use,
 as many of the data gathered exist across very dynamic, diverse, and
 large partitions. 
As such, the scalability of storing and accessing big data must increase
 with it, relevant to the structures and locations of these data repositories. 
Developing a database in an ecosystem - Hadoop - that supports tools of
 two major concepts for achieving scalability within big data -
 data-parallelism and task-parallelism - our team has built a data
 warehousing solution to structure and store the data based on both
 optimized and normalized schema designs, drafted from entity-relationship
 models designed with the intention of enabling rapid storing and accessing
 through the Hadoop MapReduce process \cite{BigDataComputing}. 
Through a combination of these systems and data storage within
 Amazon Simple Storage Service (S3) and Apache Hadoop's native Hadoop
 Distributed File System (HDFS), we analyze performance between two
 approaches toward data- and task-parallelism using data gathered from
 the stock market.

\subsection{Apache Hadoop Ecosystem}

Motivated through the opportunity within big data and parallel computing,
 which enables massive amounts of data to be rapidly accessed for complex
 analysis and distributed across a scalable, cost-effective distribution of
 servers, we aligned our project with a modern database application, Hadoop,
 which provides a data lake ``ecosystem'' supporting both task- and
 data-parallelism across the many software applications within the Apache
 suite in addition to software that can be managed and accessed within
 a network of servers, called a cluster \cite{Intel, BigDataComputing}.

\subsection{Hadoop MapReduce}

The cluster of server nodes enables users to read data from the same source,
 simultaneously, as the data at that source is partitioned and
 processed across multiple servers – a master and at least one slave 
 – through leveraging a processing framework called MapReduce to assign
 nodes for ``mapping'' and ``reducing'' by applying various configurations,
 such as related to memory allocation to and volumes of mappers and reducers. 
As data is mapped, it is reprocessed into a derivative data set,
 split into tuples that are then processed across the cluster
 of server nodes, in parallel, and reassembled in the reduce process
 from which it is delivered to the end-user, whether it be Enterprise
 Resource Planning (ERP) or a personal user running a SQL “SELECT” statement.

\subsection{Application Integration and Data Ingestion}

The parallelizability of Hadoop is central to this study as the primary
 objective is to rapidly store and access financial market data for 
 buy-sell decision-making. 
The scope of applied analysis in this study is focused on the ability to
 scale and process a combination of quantitative stock market data and
 qualitative Twitter data related to the quantitative data. 

Quantitative data was gathered from a markets data vendor through an
 Application Programming Interface (API) on 15-minute intervals using R
 programming language. 
Qualitative data was gathered and processed through a Twitter API
 using Python programming language. 
The data, structured and stored into a Hive data warehouse is created and
 managed using Hive Query Language (HQL) stored both locally and
 within an Amazon S3 storage bucket. 
R will be used via Open Database Connectivity (ODBC) to access and
 build proof-of-concept models using the data. Once developed,
 Python will be deployed within the data lake to process and
 derive predictive data through machine learning decisions.

\subsection{Big Data System Implementation}

In order to provide manageable storage repositories that scale 
 well to size and data diversity, 
 we have implemented our solution using S3 and HDFS. 
S3 and HDFS are designed for large sets of data. 
Therefore, integrity of data and ingestion systems are able
 to be well maintained. 
This is essential in an environment that may need to support 
 many simultaneous users, each with different data needs. 
The HDFS implementation in this project is housed across three 
 - one master and two slave - Amazon EC2 servers 
 whereas S3 is a standalone repository within Amazon's cloud suite.

Additionally, Apache Hive is used as a data warehouse 
 because it is operated with HQL, 
 which is easily communicable for SQL users. 
Furthermore, Hive allows for storage of massive databases tables and
 is well-suited for big data application integration, 
 including the Apache software suite, of which Hadoop is a product. 
The data warehouse graphical user interface is provided by Cloudera Hue.

\subsection{Data Warehouse Schema and Selection}

Two schemas were designed for the warehouse in a star configuration.
This first schema was designed in a full normalized fashion,
 which is known as a snowflake schema.
The second schema was based on the snowflake schema, 
 but denormalized to limit the number of tables, 
 which limits the number of joins required in queries.
Query performance will be measured on these schemas to determine
 which design will be better for this use case. 


\subsection{Performance Metrics and Evaluation}

Performance is based on speed taken to process our 3 gigabytes of data 
 - once processed into their respective storage systems - 
 into the Hive data warehouse, which includes table creation and loading. 
We will use a repeated measures analysis and a one-tailed hypothesis 
 test to determine optimal performance among S3 and HDFS groups, 
 which is then used to compare and assess benchmarks between the 
 combinations of MapReduce settings and database schemas among the
 best-performing S3 and HDFS configuration.




\section{Data}

This study investigates data warehouse models for housing stock
 price data and semi-structured alternative data. 
Both daily and intraday stock price data were collected
 for use in this analysis. 
Twitter was chosen as the primary source of alternative data because of
 ease-of-access to the \href{https://developer.twitter.com/en/docs}{Twitter API} 
 and the large volume of available data.
The main features of the collected data are summarized in Table ~\ref{DataFeatures}.

\begin{table}
	\renewcommand{\arraystretch}{1.3}
	\caption{Data Features}
	\label{DataFeatures}
	\centering
	\begin{tabular}{c|l}
		\hline
		Source       & Features\\
		\hline
		\hline
		Stock Daily  & Prices: High, Low, Open, Close\\
		\hline
		\multirow{10}{*}{Stock Intraday} &  Prices: High, Low, Open, Close \\
		&  High Bollinger bands\\
		&  Mid Bollinger bands\\
		&  Low Bollinger bands\\ 
		&  Nominal moving average\\
		&  Historical moving average\\
		&  Signal moving average\\ 
		&  Exponential moving average\\
		&  Stochastic 5-day indicator (Slow K)\\
		&  Stochastic 3-day indicator (Slow D)\\
		\hline
		Tweets       & Text, URLs, Hashtags, Mentions, Users\\
		\hline
	\end{tabular}
\end{table}

\subsection{Stock Data}

The stock price data was collected through an API provided by
 \href{https://www.alphavantage.co/}{Alpha Vantage}.
This API provided access to daily prices, intraday stock prices,
 and intraday price features.
Intraday stock data contained 35 features sampled at 15 minute intervals. 
The intraday features were from the following categories Bollinger bands, 
 stochastic oscillators, moving averages, and exponential moving averages.

Approximately 1 million rows of data were collected,
 which occurred from Oct. 04, 2019 to Oct. 24, 2019.
During the collection process, the data were recorded in files
 organized by day and category and stored in an S3 bucket.
At the end of the collection process, the files for each
 category were combined and pushed to the HDFS datalake.

\subsection{Twitter Data}

Messages on Twitter (called tweets) are mainly comprised of
 tweet ID, timestamp, author (screen name), and text.
Tweets can also contain many other features such as
 URLs, hashtags, emojis,  and mentions.
For this analysis, in addition to the main features,
 URLs, hashtags, and mentions were also collected in the data warehouse.
A mention is a reference to another Twitter user's screen name
 in the text of a tweet.
A hashtag is some collection of characters starting with
 '\#' without white space.
Generally, the character portion of a hashtag will be a word or set of words,
 but this is not necessary.

Approximately 300,000 tweets were collected from over 100 Twitter users.
The data from Twitter is returned in JavaScript Object Notation (JSON).
The features of interest were extracted from the JSON files and
 reformatted in tab separated files (TSV).
After the tweets were extracted from Twitter,
 mentions of company names or stock symbols were extracted from the tweet text by
 matching sections of tweet text to company names and company stock symbols.
Once the data was collected and processed, it was pushed to the HDFS datalake.


\section{Data Warehouse Development}

Data warehouses are often conceptually designed with a \textit{star} schema 
 \cite{BuildingtheDWCH11}.
In the star design, there is a central table (called the fact table),
 which contains the unifying features of the dataset and keys to other tables
 \cite{BuildingtheDWCH11}.
The tables surrounding the fact table (called dimension tables) 
 contain information related to a category of the facts \cite{Enterprise}.
In the dataset for this analysis, the unifying features are timestamp and
 company stock symbol, thus the fact table contains these features
 and several other descriptive features related to the companies.
The natural dimensions of the fact table are intraday stock data,
 daily stock data, and tweet data.
In some cases, dimensions of the facts exhibit an
 inherent hierarchical structure. 
When the dimensions of a star schema are broken out into the tables
 that make up the hierarchical structure of the dimensions, the schema
 is in its \textit{snowflake} form \cite{WarehouseDesignApproaches}.
In this study, two schemas were designed: one in snowflake form and
 the other in a more optimal form for query speed.

\subsection{Snowflake Schema}

\begin{figure}
	\centering
	\includegraphics[width=2.5in]{Snowflake_Conceptual_Schema.png}
	\caption{Conceptual Diagram of the Data Warehouse Snowflake Schema}
	\label{snowflake}
\end{figure}

As noted previously, the fact table of this star design contained the
 company information and timestamps.
The primary key of the fact table is a composite key that includes
 stock symbol and timestamp.
There are three natural dimensions of fact table: daily stock data,
 intraday stock data, and tweet data.

Both daily and intraday stock data are naturally keyed by the
 combination of timestamp and stock symbol,
 thus naturally keyed to the fact table.
The stock data was spilt into two dimensions: daily data and intraday data.
The daily data comes in a form suitable for the snowflake design.
However, the intraday data was normalized into five tables for the 
 snowflake design: prices, Bollinger bands, moving averages, exponential
 moving averages, and stochastic indicators.
The integration of the fact table and the normalized stock table is shown
 conceptually in Fig. \ref{snowflake} (left side of the schema diagram).

Unlike the stock data, the tweet data did not naturally join to the fact
 table of the schema.
Like the stock data, the tweet data came with a timestamp, but stock symbol
 is not a nominal feature of tweets.
The stock symbol feature was generated by extracting matching strings
 during the data collection process; therefore, stock symbols are not
 guaranteed to be non-null in the tweet dimension.
Additionally, the combination of timestamp and stock symbol is not guaranteed
 to be a primary key into the tweet table for tweets with non-null stock symbol
 fields.
However, Twitter assigns a tweet ID for each tweet, which is guaranteed to be
 a primary key.
Thus, the same features can still be used to join the tweet dimension to the 
 fact table, but cannot serve as a primary key.
As noted previously, three secondary features of tweets are used in this study.
Each of these features, mentions, hashtags, and URLs, represents a hierarchical
 member of the tweet dimension.
Tables were created for each of these features to follow the snowflake design.
These additional tables required join tables because there is a many-to-many
 cardinality between the tweet dimension and each of the three secondary members.
The result of the tweet dimension normalization is shown conceptually
 in Fig. \ref{snowflake} (right side of schema diagram).

\subsection{Denormalized Star Schema}

While the snowflake schema is suitable for general use cases.
The number of tables should be limited to decrease query time to support
 fast data transfer to a machine learning system.
The number of intraday tables and tweet tables were reduced to support
 fast query times.
All intraday tables were combined into one table. 
Each secondary member of the Twitter dimension hierarchical structure was
 subsumed into a copy of the main tweet table, reducing the number of tables
 in the tweet dimension from seven to three.
The schema resulting from this denomalization process is shown
 conceptually in Fig. \ref{star}.


\begin{figure}
	\centering
	\includegraphics[width=2.5in]{Star_Conceptual_Schema.png}
	\caption{Conceptual Diagram of the Data Warehouse 
		Denormalized Star Schema}
	\label{star}
\end{figure}

\section{Implementation}

Before we delve into the details of dimensional modeling, it's helpful to focus on the technology, tool and technique that we will use to collect, process the data and conduct our analysis. Multiple vendors provide the platform and infrastructure for these technologies, for this project we will use Amazon Web Service (AWS)

AWS is a set of cloud computing services provided by Amazon that are accessible over the internet, it’s pretty much consist of a collection of compute and storage system which operate at a high scale. AWS provides multiple services based on the user application. For our project going to use the Elastic MapReduce (EMR) service.

AWS EMR is a set of pre-configured infrastructures with the latest software and tool required for Big Data analysis and processing. EMR is based on Apache Hadoop, Hadoop Mapreduce, Hadoop Elastic Computer Cloud(EC2) , Amazon Simple Storage Service (S3), HUE and Hive. Fig. \ref{EMR} shows the ecosystem of EMR ecosystem.

Hadoop it’s open source software framework written in java for the distributed storage and distributed processing of a very large datasets.

The core of Apache Hadoop consists of a storage part: Hadoop Distributed File System (HDFS) and a processing part (Hadoop MapReduce):

\begin{figure}
	\centering
	\includegraphics[width=2.5in]{EMR_Ecosystem.png}
	\caption{EMR Ecosystem}
	\label{EMR}
\end{figure}

\subsection{Hadoop Distributed File System (HDFS)}

Not like traditional file system, HDFS is a distributed file system designed to run on commodity hardware. One of the advantage of that is the highly fault tolerant and deployed on low cost hardware. HDFS it's a master (Namenode)/slave(Datanodes) architecture. Where the NameNode is the controller which consist storing data to the DataNode and also store metadata of the data which include Namespace or lookup table used to locate each file from the multiple DataNode (Servers). The Fig. \ref{HDFS} shows the Architecture of HDFS \cite{HDFS}.

It’s a programming model for a large scale data processing. With the high demand of computing power, computer architecture evolved over the year from serial computing to parallel computing where tasks are distributed and executed simultaneously across multiple processors within the same computer. Now with the mapReduce instead of using one instance with multiple processing power, we will use multiple computers with multiple processing power, this is called distributed computing system. This is actually what enable us to manage and process

vast amounts of data (multi-terabyte data-sets) in-parallel on large clusters (thousands of nodes) of commodity hardware in a reliable, fault-tolerant manner. A MapReduce framework consists of two workers: the Mapper and the Reducer, so how does it work? It's using the divide and conquer technique, where the input is divide into a set of small task, each task is identify by the key -value pair, where the key serve as task ID and the value the is the task output. Each task is then process and execute by the mapper and the outputs are process and merge by the reducer. The Fig. \ref{EMR} show the simplified MapReduce work flow.

\begin{figure}
	\centering
	\includegraphics[width=2.5in]{HDFS_Arch.png}
	\caption{HDFS Architecture}
	\label{HDFS}
\end{figure}

\subsection{Hadoop Elastic Computer Cloud (EC2)}

From the above we may wonder how does the data are processed and the tasks executed, the EC2 is an entity of the cloud computer system which is used to for computing operation from each task.

\subsection{Amazon Simple Storage Service (S3)}

As the name imply, the S3 is a storage systems, which is used to store and retrieve any amount of data any time, from anywhere on the web. It's reliable, fast, and inexpensive.

\subsection{Apache Hive}

Apache Hive is a data warehouse infrastructure built on top of Hadoop. Hive enable us to structure, organize, model the data and query the data using SQL like language called HiveQL.

\subsection{Cloudera Hue}

Hadoop Hue is an open source web application that served as user interface for Hadoop components. The user access Hue right from within the browser and can interact and with the Hadoop ecosystem such as HDFS and MapReduce applications. Users do not have to use command line interface to use Hadoop ecosystem if he will use Hue.

\section{Results}

Maybe a table comparing read times for the different options

\begin{itemize}
	\item Snowflake vs Denormalized Star
	\item S3 vs HDFS
	\item HDFS mapreduce settings?
\end{itemize}

\section{Analysis}

A statistical analysis of the results

\begin{itemize}
	\item Discuss sample size of measures
	\item Looks like a repeated measures analysis
\end{itemize}


\section{Conclusion}
\textbf{EXAMPLE CONCLUSION:}
The scalability of large sets of data is of ever-increasing importance in the data community, which itself is ever-increasing with the advancement of modern technology. Modern technology itself is enabling both an increase in both data volume as well as data diversity being captured and stored. Consequently, parallelism is of utmost importance in making use of this data. Herein analyzed are methods that are proven within the scope of big data. As modern technology continues to advance, these parallel principals will serve as the root from which methodologies will be developed. As with the businesses and industries the data captured will serve, there will remain a trade-off that must be assessed for each business model. This study has provided two primary storage methods - S3 and HDFS - in addition to different data warehousing schema design - normalized versus optimized, star versus snowflake - and data distribution - herein, MapReduce - techniques. Through analysis using repeated measures, S3 outperforms HDFS when X, Y, Z and HDFS outperforms S3 when Z, Y, X. Furthermore, MapReduce configurations are constant across both storage methods - HDFS and S3 - with an optimized star schema performing faster, but a normalized snowflake schema performing more reliably within a conventional three-tiered system architecture.

% Can use something like this to put references on a page
% by themselves when using endfloat and the captionsoff option.
\ifCLASSOPTIONcaptionsoff
  \newpage
\fi

\begin{thebibliography}{1}

\bibitem{BigDataComputing}
R. Kune, P. Konugurthi, A. Agarwal, R. Chillarige, R. Buyya,
 "The Anatomy of Big Data Computing," Software: Practice and Experience,
 Vol. 46 no. 1, pp.79-105, Jan. 2016. 

\bibitem{BuildingtheDWCH11}
W. H. Inmon, "Unstructured Data and the Data Warehouse," in 
  \emph{Building the Data Warehouse},
  4th ed. Hoboken: Wiley, 2005, ch. 11.
  Accessed on Nov. 6, 2019 [Online]. 
  Available: \\ https://learning.oreilly.com/library/view/building-the-data/9780764599446

\bibitem{WarehouseDesignApproaches}
I. Moalla, A. Nabli, L. Bouzguendam and M. Hammami,
 "Data warehouse design approaches from social media: review and comparison,"
 Social Network Analysis and Mining., Vol. 7, no. 1, pp. 1-14, Jan. 2017.
 Accessed on: Nov. 6, 2019 [Online]. 
 Available doi: 10.1007/s13278-017-0423-8

\bibitem{Enterprise}
A. Gorelik, "Historical Perspectives," in 
 \emph{The Enterprise Big Data Lake},
 1st ed. Sebastopol, CA: Wiley, 2019, ch. 2, pp. 25-47.

\bibitem{Intel}
"Extract, Transform, and Load Big Data with Apache Hadoop," Intel, USA, 2013.
 Available:\\ https://software.intel.com/sites/default/files/article/402274/etl-big-data-with-hadoop.pdf

\bibitem{HDFS}
D. Borthakur, HDFS Architecture Guide, The Apache Software Foundation,
 August 22 2019. Accessed on: Nov. 8, 2019. [Online] Available: \\
 https://hadoop.apache.org/docs/r1.2.1/hdfs\_design.html

\end{thebibliography}


% that's all folks
\end{document}


